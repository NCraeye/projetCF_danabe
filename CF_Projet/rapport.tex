\documentclass[a4paper]{book}
\usepackage{fullpage}

\usepackage[utf8]{inputenc}
\usepackage[T1]{fontenc}
\usepackage[francais]{babel}

\usepackage{latexsym}
\usepackage{fancyhdr}
\usepackage{makeidx}
\usepackage{graphics}
\usepackage{graphicx}
\usepackage{longtable}
\usepackage{moreverb}
\usepackage{listings}

\newcommand{\altarica}{{\sc AltaRica}}

\begin{document}

\title{Master 1, Conceptions Formelles\\
Projet du module \altarica\\
Synthèse (assistée) d'un contrôleur du niveau d'une cuve}

\date{}

\author{Craeye Nathalie \and Faltrept Bérénice \and Lejeune David}

\maketitle

\chapter{Le sujet}
\input{tank}

\chapter{Le rapport}
\section{Rôle de la constante {\tt nbFailures} (2 points)}

$nbFailures$ est la constante qui limite le nombre de vannes pouvant tomber en panne simultanément. Elle permet de lancer une configuration du système avec un nombre de pannes prédéfini. \\
L'assert du composant System assure, lors du lancement du modèle par l'utilisateur, que ce dernier a une configuration qui fait sens : le nombre de pannes étant limité au nombre de vannes existantes et étant en panne dans le modèle en cours.

$ValveVirtual$ est le composant permettant de simuler une valve parfaite, autrement dit sans panne. En l'intégrant au modèle, via CtrlVV, on peut alors confirmer la cohérence du modèle indépendemment de la gestion des pannes. \\
Pour chaque niveau d'itération, on va donc créer une version virtuelle, sans panne, pour contrôler le système mais aussi créer la version avec Ctrl et Valve qui assure la gestion des pannes.

On retrouve cette organisation dans le fichier GNUmakefile, à la racine du projet. Il lance toutes les configurations possibles du système sur lequel nous travaillons. On peut le voir lors de la génération du fichier .time : une première boucle assure la génération des fichiers pour tous les contrôleurs ( Ctrl et CtrlVV). Pour chacun d'entre eux, une boucle va produire des fichiers associés aux nombres de pannes.

\section{Résultats avec le contrôleur initial {\tt Ctrl}}

\subsection{Calcul d'un contrôleur}

\subsubsection{Avec 0 défaillance (1 point)}
\lstinputlisting{Res/System0FCtrl.res}
\lstinputlisting{Res/System0FCtrl0F1I.res}
\lstinputlisting{Res/System0FCtrl0F2I.res}
\lstinputlisting{Res/System0FCtrl0F3I.res}
\lstinputlisting{Res/System0FCtrl0F4I.res}
\paragraph{Interprétation des résultats}

On observe que sur les différentes itérations : excepté la première, elles sont toutes identiques. De plus, notre système n'a aucune deadlock et les situations redoutées n'existe plus après cette première tentative. Donc les modifications appliquées à partir de l'itération 1 sont concluantes et permettent d'obtenir un système que l'on peut supposer fiable.
\begin{itemize}
%% TODO : comprendre comment faire et répondre posément
	\item oui. À partir de l'itération 1, il n'y a plus de situation critique et nous n'avions aucun deadlock en tout cas. 
	\item Avec un débit aval de 0, on obtient moins de sommets. On présume donc qu'il est optimal d'avoir un débit de 0 puisque cela nous permet de n'avoir aucune situation critique en gérant moins de cas.

\end{itemize}
\subsubsection{Avec 1 défaillance (1 point)}
\lstinputlisting{Res/System1FCtrl.res}
\lstinputlisting{Res/System1FCtrl1F1I.res}
\lstinputlisting{Res/System1FCtrl1F2I.res}
\lstinputlisting{Res/System1FCtrl1F3I.res}
\lstinputlisting{Res/System1FCtrl1F4I.res}
\paragraph{Interprétation des résultats}

\subsubsection{Avec 2 défaillances (1 point)}
\lstinputlisting{Res/System2FCtrl.res}
\lstinputlisting{Res/System2FCtrl2F1I.res}
\lstinputlisting{Res/System2FCtrl2F2I.res}
\lstinputlisting{Res/System2FCtrl2F3I.res}
\lstinputlisting{Res/System2FCtrl2F4I.res}
\paragraph{Interprétation des résultats}

\subsubsection{Avec 3 défaillances (1 point)}
\lstinputlisting{Res/System3FCtrl.res}
\lstinputlisting{Res/System3FCtrl3F1I.res}
\lstinputlisting{Res/System3FCtrl3F2I.res}
\lstinputlisting{Res/System3FCtrl3F3I.res}
\lstinputlisting{Res/System3FCtrl3F4I.res}
\paragraph{Interprétation des résultats}

\subsection{Calcul des contrôleurs optimisés (2 points)}

\section{Rôle des composants {\tt ValveVirtual} et {\tt CtrlVV} (4 points)}

\section{Résultats avec le contrôleur initial {\tt CtrlVV}}

\subsection{Calcul d'un contrôleur}

\subsubsection{Avec 0 défaillance (1 point)}
\lstinputlisting{Res/System0FCtrlVV.res}
\lstinputlisting{Res/System0FCtrlVV0F1I.res}
\lstinputlisting{Res/System0FCtrlVV0F2I.res}
\lstinputlisting{Res/System0FCtrlVV0F3I.res}
\lstinputlisting{Res/System0FCtrlVV0F4I.res}
\paragraph{Interprétation des résultats}

\subsubsection{Avec 1 défaillance (1 point)}
\lstinputlisting{Res/System1FCtrlVV.res}
\lstinputlisting{Res/System1FCtrlVV1F1I.res}
\lstinputlisting{Res/System1FCtrlVV1F2I.res}
\lstinputlisting{Res/System1FCtrlVV1F3I.res}
\lstinputlisting{Res/System1FCtrlVV1F4I.res}
\paragraph{Interprétation des résultats}

\subsubsection{Avec 2 défaillances (1 point)}
\lstinputlisting{Res/System2FCtrlVV.res}
\lstinputlisting{Res/System2FCtrlVV2F1I.res}
\lstinputlisting{Res/System2FCtrlVV2F2I.res}
\lstinputlisting{Res/System2FCtrlVV2F3I.res}
\lstinputlisting{Res/System2FCtrlVV2F4I.res}
\paragraph{Interprétation des résultats}

\subsubsection{Avec 3 défaillances (1 point)}
\lstinputlisting{Res/System3FCtrlVV.res}
\lstinputlisting{Res/System3FCtrlVV3F1I.res}
\lstinputlisting{Res/System3FCtrlVV3F2I.res}
\lstinputlisting{Res/System3FCtrlVV3F3I.res}
\lstinputlisting{Res/System3FCtrlVV3F4I.res}
\paragraph{Interprétation des résultats}

\subsection{Calcul des contrôleurs optimisés (2 points)}

\section{Conclusion (2 points)}

\end{document}
